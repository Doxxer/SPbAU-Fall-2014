\documentclass[russian,table]{article}
\usepackage[table]{xcolor}
\usepackage[T1]{fontenc}
\usepackage[utf8]{inputenc}
\usepackage{geometry}
\geometry{verbose,tmargin=1cm,bmargin=2cm,lmargin=1cm,rmargin=1cm}
\usepackage{amsmath}
\usepackage{float}
\usepackage{tikz}
\usetikzlibrary{automata,positioning}
\usepackage{textcomp}
\usepackage{amssymb}
\usepackage{graphicx}
\usepackage{babel}
\usepackage{mathtools}
\usepackage{color}
\usepackage[T2A]{fontenc}
\usepackage{listings}
\usepackage{slashbox}
\usepackage{multirow}
\makeatletter
\@ifundefined{date}{}{\date{}}
\begin{document}

\title{Формальные грамматики. HW\#2}
\author{Тураев Тимур, SPbAU, SE, 604 group}
\maketitle

\paragraph*{7.}

\textit{Для произвольной данной линейной грамматики $G$, пусть $L=\{vu \mid uv \in L(G)\}$ — циклический сдвиг порождаемого ею языка. Построить алгоритм, определяющий принадлежность данной на входе строки языку $L$, и использующий как можно меньше памяти. \\}

\textit{Некоторые мысли и идеи по этой задаче.}

Можно запустить распознавание каждой строки из всех возможных циклических сдвигов данной строки. И если хотя бы одна строка принадлежит языку, то ответим <<да>>. Нужно научиться делать 2 вещи: проверять строку на принадлежность языку $L(G)$ и как-то организовать перебор всех линейных сдвигов строк.
Первое сделать просто: это алгоритм 10.1 из 14 лекции, распознающий обычные языки за $O((\log n)^2)$ памяти.
Второе: можно добавить параметр в алгоритм, который характеризует текущий сдвиг $\mathit{offset}$, и в том месте где идет обращение к символам входной строки будем обращаться по индексу $1 + ((i + \mathit{offset}-1) \mod n)$, где $i$ -- индекс символа в рассматриваемой нами строке.


\end{document}